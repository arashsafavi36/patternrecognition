
\subsection*{task 1.3 [10 points] \\[1ex] implementing the QR algorithm}

Given the matrix $\mat{C}$ as computed in task 1.1, implement the following iterative procedure

\vspace{1ex}

\hrule
\begin{algorithmic}
\State $\mat{C}_0 = \mat{C}$
\State
\For{$t = 1, \ldots, 10$}
\State
\State compute the QR decomposition of $\mat{C}_{t-1}$ to obtain matrices $\mat{Q}_t$ and $\mat{R}_t$
\State
\State given matrices $\mat{Q}_t$ and $\mat{R}_t$, compute
\begin{equation*}
\mat{C}_t = \mat{R}_t \mat{Q}_t
\end{equation*} 
\EndFor
\end{algorithmic}
\hrule
\vspace{1ex}
%%%%%
%%%%%
%%%%% enter your code into the following environment
%%%%%
%%%%%
\begin{python}
# past your code here


\end{python}
%%%%%
%%%%%
%%%%%
%%%%%
%%%%%
\vfill

Run your code and have a look at the diagonal entries of the resulting matrix $\mat{C}_{10}$. Compare them against the spectra you computed above; what do you observe ?
\color{blue} \\[1ex]
%%%%%
%%%%%
%%%%% enter your discussion here
%%%%%
%%%%%
enter your discussion here \ldots
%%%%%
%%%%%
%%%%%
%%%%%
%%%%%
\color{black}
\vspace{2cm}

Now, consider the following equivalencies
\begin{equation*}
\mat{C}_t = \mat{R}_t \mat{Q}_t = \inv{\mat{Q}_t} \mat{Q}_t \mat{R}_t \mat{Q}_t = \mat{Q}_t^{-1} \mat{C}_{t-1} \mat{Q}_t = \trn{\mat{Q}_t} \mat{C}_{t-1} \mat{Q}_t 
\end{equation*}
Can you use this insight to explain the result you get form running the QR algorithm ?
\color{blue} \\[1ex]
%%%%%
%%%%%
%%%%% enter your discussion here
%%%%%
%%%%%
enter your discussion here \ldots
%%%%%
%%%%%
%%%%%
%%%%%
%%%%%
\color{black}
\newpage



\subsection*{bonus [5 points]}

If you want to impress your instructors, perform run-time measurements for your implementation of the QR algorithm as well.
\color{blue} \\[1ex]
%%%%%
%%%%%
%%%%% enter your discussion here
%%%%%
%%%%%
enter your discussion here \ldots
%%%%%
%%%%%
%%%%%
%%%%%
%%%%%
\color{black}



