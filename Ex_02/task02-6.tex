
\subsection*{task 2.6 [10 points] \\[1ex] the QR algorithm can sort!}

In task 1.3, you already implemented the QR algorithm and used it to compute the eigenvalues of a sample covariance matrix. Yet, amazingly (?), the QR algorithm can do much more \ldots

To see an example, implement the following procedure in Numpy / Scipy


\begin{enumerate}
\item Assume you were given a vector
\begin{equation*}
\vec{x} = \trn{[x_1, x_2, x_3, \ldots, x_n]}
\end{equation*}
consisting of $n$ elements $x_i \in \mathbb{R}$
\begin{python}
# get the dimension of the input vector
n = x.shape[0]
\end{python}

\item Create an $n \times n$ tri-diagonal matrix $\mat{X}$ whose main diagonal contains the elements of $\vec{x}$ and whose off diagonals contain a fixed small number $\epsilon$. In other words, create a matrix like this
\begin{equation*}
\mat{X} = 
\begin{bmatrix}
     x_1 & \epsilon \\
\epsilon &      x_2 & \epsilon \\
         & \epsilon &      x_3 & \epsilon \\
         &          &          &   \ddots & \epsilon \\ 
         &          &          & \epsilon &      x_n
\end{bmatrix}
\end{equation*}
\begin{python}
# build the tri-diagonal matrix with the vector x on the main
# diagonal and eps on both off-diagonals
off_diag = [eps] * (n-1)
X = np.diag(x) + np.diag(off_diag, k=1) + np.diag(off_diag, k=-1)
\end{python}

\item compute 
\begin{equation*}
\mat{X}' = \exp \bigl( \mat{X} \bigr)
\end{equation*}
\begin{python}
# compute the matrix exponential
X_prime = la.expm(X)
\end{python}

\item feed $\mat{X}'$ into the QR algorithm to obtain a matrix
\begin{equation*}
\mat{Y}' = \operatorname*{qrAlgorithm} \bigl( \mat{X}' \bigr)
\end{equation*}
\begin{python}
# apply the qr-algorithm
Y_prime = qr_alg(X_prime, **kwargs)
\end{python}

\item compute 
\begin{equation*}
\mat{Y} = \log \bigl( \mat{Y}' \bigr)
\end{equation*}
\begin{python}
# compute the matrix logarithm
Y = la.logm(Y_prime)
\end{python}
\end{enumerate}

\textbf{NOTE:} In this task, we write $\exp(\cdot)$ and $\log(\cdot)$ to denote the matrix exponential- and logarithm functions. Scipy provides them as \keyword{la.expm} and \keyword{la.logm}.

In order to work with a specific example, run you code with the following input and parameter
\begin{align*}
\vec{x}  & = \trn{[ 4, -3, 2, 7, 12, 1 ]} \\
\epsilon & = 0.0001
\end{align*}

Run your code four times using $t_{\text{max}} \in \{ 1, 5, 10, 50 \}$ iterations for the QR algorithm in step 4. After each run, print the diagonal entries of the matrix $\mat{Y}$ you obtain in step 5. What do you observe ?

\color{blue} \\[1ex]
%%%%%
%%%%%
%%%%% enter your discussion here
%%%%%
%%%%%

Just for completeness we insert the full code here:

\color{black}
\begin{python}
def qr_alg(A:np.ndarray, k:int =10) -> np.ndarray:
    # iterate for k iterations
    for _ in range(k):
        # decompose and build next matrix
        Q, R = la.qr(A)
        A = R @ Q
    # return last
    return A
def qr_sort(x:np.ndarray, eps:float =1e-4, **kwargs) -> np.ndarray:
    # get the dimension of the input vector
    n = x.shape[0]
    # build the tri-diagonal matrix with the vector x on the main
    # diagonal and eps on both off-diagonals
    off_diag = [eps] * (n-1)
    X = np.diag(x) + np.diag(off_diag, k=1) + np.diag(off_diag, k=-1)
    # compute the matrix exponential
    X_prime = la.expm(X)
    # apply the qr-algorithm
    Y_prime = qr_alg(X_prime, **kwargs)
    # compute the matrix logarithm
    Y = la.logm(Y_prime)
    # return the diagonal entries
    return np.diag(Y)
# create some sample input
x = np.asarray([4, -3, 2, 7, 12, 1], dtype=np.float32)
# apply for different iterations of the qr algorithm
for k in [1, 5, 10, 50]:
    sorted_x = qr_sort(x, k=k)
    sorted_x = np.rint(sorted_x)
    print("k=%i:\t" % k, sorted_x)
\end{python}
\color{blue}

The output looks as follows:

$k=k:  [ 4, -3,  2,  7, 12,  1]$ \\
$k=5:  [ 4, 12,  7,  2, -3,  1]$ \\
$k=10: [12,  4,  7,  2, -3,  1]$ \\
$k=50: [12,  7,  4,  2,  1, -3]$

One can observe that with increasing number of iterations of the QR algorithm, the output is being sorted better and better.

%%%%%
%%%%%
%%%%%
%%%%%
%%%%%
\color{black}


